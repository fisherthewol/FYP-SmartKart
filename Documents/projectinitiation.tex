\documentclass[11pt, a4paper, notitlepage]{report}
% Bibliography stuff to do later.
%\usepackage[authoryear]{natbib}
%\bibliographystyle{agsm}
\usepackage[utf8]{inputenc} % Ensures we can use UTF8.
\usepackage{hyperref} % Must come last; allows for links.
\hypersetup{
    hidelinks
}
% Details.
\title{Smart Kart (Project Initiation Document)}
\date{November 2021}
\author{George Jacob Anthony Kokinis}

\begin{document}

\maketitle
\begin{center}
    Student Number 201910280
    
    Word Count: XXXX % You must also replace “XXXX” with the actual word count 
    %(excluding acknowledgements, abstract, table of contents, references and 
    %appendices) of your document.
\end{center}
\newpage

\tableofcontents

\chapter{Project Background and Purpose}
\section{Objectives}
The usefulness of this project is the safety benefits it may provide. If the 
Artefact produced can reduce the frequency that a driver needs to check their 
speedometer, then they will be more aware of road conditions.

This project is motivated by the increasing deployment of average speed check 
zones throughout the UK, and the potential they have for leading to distracted 
driving.
% Source?? Can't find source on "increasing" deployment.

\subsection{Primary Objectives}\label{subsec:PrimaryObjectives}
The Primary Objectives of this project are as follows:
\begin{itemize}
    \item An app (for a smartphone) that determines when the user is in an 
    average speed check zone, and begins tracking average speed.
    \item Said app then warns, with an audible warning, if the average speed is 
    above the limit.
    \item Voice commands may be used to launch the app at the user's request 
    (for example, if there is temporary speed check area).
\end{itemize}

\subsection{Secondary Objectives}
If time allows, the project may achieve the following objectives:
\begin{itemize}
    \item The app allows for the user to manually set what the speed limit is 
    (for example if a road has a temporarily reduced speed limit).
    \item The app allows for setting the audible warning to a custom sound.
\end{itemize}

\subsection{Tertiary Objectives}
In many territories globally, the use of devices to detect speed cameras is 
illegal, and apps are either explicitly or potentially illegal %(france code de 
%la route).
It would be ideal if the app could detect it was in such a territory, and 
prevent its own usage.

As well, the app may allow the users to contribute to an open dataset of 
average speed checks.

\section{Scope}\label{sec:Scope}
There is scope within the project to provide an app that is cross platform, 
using technologies such as Flutter, UNO, and .NET MAUI (née 
Xamarin.Forms)\footnote{Flutter (citation), UNO (Citation), .NET MAUI 
(Citation).}. This would allow users of the major mobile ecosystems to use the 
app. 

However, doing so may hamper the ability of the developer to write 
sufficient code in the time provided. Hence, a cross-platform app is not in 
scope.

There is also scope to build an independent database of average speed check 
areas. This would be a significant undertaking, and there are already public 
datasets available for use. As such, making such a database is not in scope. 
However, allowing for contribution to such datasets from within the app would 
serve the public good, and may well be within the scope of the project.
% source which datasets?

\section{Deliverables}
The Project will deliver an Android Application that meets the Primary 
Objectives (defined in section \ref{subsec:PrimaryObjectives}).

The project will have met its objectives when the app matches provides features 
defined by the Primary Objectives. It will exceed them if it \textbf{also} 
implements Secondary or Tertiary Objectives.

\section{Constraints}
Testing the application in the most \textit{straightforward} way may be 
constrained by applicable law, insurance, and ethics. Otherwise, there are no 
external constraints known at this time.

\section{Assumptions}
There are currently no unknowns for this project; hence, there are no 
assumptions to be made.

\chapter{Project Rationale and Operation}
\section{Project Benefits}
Successful delivery of this project will most benefit drivers (primarily in the 
UK, but possibly in other territories), with phones, who often drive in areas 
with average speed check zones. 

By reducing the time spent monitoring their speedometer, they can be more 
conscious of the road around them, and be more alert to potential incidents.

As well, by providing \textit{advance} warning of exceeding the average speed, 
they may be able to drive in a smoother manner and avoid the kangaroo-ing" 
effect; this can reduce congestion and prevent accidents.
% Sources?

\section{Project Operation}\label{sec:projectOperation}
Agile Software Development is a paradigm, commonly defined by the Manifesto 
(source); within this paradigm are many varied methodologies. For projects 
where there is only one team member, methodologies such as Kanban, 
Scrum (as modified by Scrum for One (source)), and Extreme Programming are 
optimal. (source!!).

Kanban is the most suitable methodology for the software development in this 
project...% - why?????

The overall planning and schedule for this project is laid out in the Gantt 
Chart in section \ref{subsec:schedGanttCt}. The intention is to stick with this 
schedule; however, the agile paradigm requires flexibility, so this may change 
with time.

\section{Options}
As discussed in the section on Scope (\ref{sec:Scope}), for developing this 
project there is certainly one choice to be made: is the app developed using 
one of various cross-platform frameworks, or using platform-native frameworks.

In that discussion, it was already decided that using a cross-platform 
framework is not in the scope; however, even within the realm of 
platform-native, there are more choices to be made.

\subsection{Which Platform?}
The Project's nature requires a mobile platform. However, there is a choice 
within this; despite Android and iOS' apparent dominance of the market (with 
(source) claiming combined a 99.19\% share, as of September 2021), there 
are various other mobile platforms, such as Tizen\footnote{Backed by the Linux 
Foundation, but primarily developed by Samsung.}, KaiOS, and Sailfish OS; as 
well as mobile implementations of desktop platforms, such as pureOS.

Despite this variety, Android stands tall as the easiest to develop for and 
most accessible platform. The Android Open Source Project means that obtaining 
an image to test on, or examining the OS' source to aid in debugging, is 
relatively easy. Hence, Android is an ideal platform to develop this project 
on. Furthermore, using Android Jetpack\footnote{Android Jetpack (citation)} 
whenever possible will provide the project with a consistent base to build upon.

In comparison, developing platform-native applications for iOS devices requires 
a MacOSX device to use Xcode, and a payment of \$99 if you wish to distribute 
your application. The other mobile platforms have significantly small market 
share.

\subsection{Which Language?}
Android applications have historically been written in Java, and the foundation 
of the OS is a JVM (originally Dalvik but later ART). The ecosystem around Java 
on Android is mature and well-documented.

However, in 2011 JetBrains\texttrademark\ announced Kotlin, a JVM language 
"having the features so desperately wanted by the developers" (source). Being a 
JVM language, it was inherently "usable" on Android; but the Android Team 
announced "first-class support" in 2017 (source) and "Android development will 
become increasingly Kotlin-first," in 2019 (source).

Hence, this project will develop the app in Kotlin (rather than Java); it is 
the contemporary language, and Google suggests that "If you’re starting a new 
project, you should write it in Kotlin" (source).

\section{Risk Analysis}
The most major risks to the Project at this time are those associated with 
Covid-19: Potential future lockdowns, mutations of the virus, supply issues, 
and potential infection of the Author. These risks can be mitigated by the 
actions of the Author, but only to an extent; it depends upon the actions of 
others as well.

There are various other risks, such as personal illness, strikes, and equipment 
breakdowns; the analysis table is in Appendix \ref{app:RiskAnalysis}.

\section{Resources Required}
There are no abnormal resources required for this project. All that is needed 
is a computer to develop on, and an android phone (or the AVD emulator) to test 
with.

If any of these were unavailable (for example, if the computer being used 
became non-functional), it would delay the project until an alternative could 
be sourced (or until the university's Labs are available).
%What resources will you need for the project?  Are any non-standard?  Are they 
%already available?  What effect will it have if they are not available or are 
%delayed, and how would you manage that?

\chapter{Project Methodology and Outcomes}
\section{Initial Project Plan}
\subsection{Tasks and Milestones}
%Present a realistic task list for the entire project, broken down to a 
%suitable level of detail.  Indicate milestones against which progress can be 
%monitored.
\subsection{Schedule Gantt Chart}\label{subsec:schedGanttCt}
%Present a Gantt chart showing a schedule for all tasks, milestones and 
%deliverables. Show dependencies amongst tasks.
\section{Project Control}
%How will you manage the project day-to-day?  How will its performance be 
%monitored? How will you judge if it has been successful?
\section{Project Evaluation}
%How will you evaluate the project’s artefacts and overall outcomes?  What user 
%evaluation will you do?

\appendix
\chapter{Risk Analysis Table}\label{app:RiskAnalysis}
%What risks might affect the outcome of your project or its stakeholders?  How 
%severe are they, and what steps will you take to mitigate against them?
\begin{tabular}{|c|c|}
	\hline
	Demo Table & Demo Table \\
	\hline
	Demo table & Hello World!!! \\
	\hline
\end{tabular}
\end{document}
