\documentclass[11pt, a4paper, notitlepage]{report}
\usepackage[authoryear]{natbib}
\bibliographystyle{agsm}
\usepackage[utf8]{inputenc}
\usepackage{hyperref}
\hypersetup{
    hidelinks
}

\title{Smart Kart (Project Initiation Document)}
\date{November 2021}
\author{George Jacob Anthony Kokinis}

\begin{document}
\maketitle
\begin{center}
    Student Number 201910280

    Word Count: XXXX
\end{center}
\newpage
\tableofcontents

\chapter{Project Background and Purpose}
\section{Objectives}
The usefulness of this project is the safety benefits it may provide. If the 
Artefact produced can reduce the frequency that a driver needs to check their 
speedometer, then they will be more aware of road conditions.

This project is motivated by the increasing deployment of average speed check 
zones throughout the UK, and the potential they have for leading to distracted 
driving.
% Source?? Can't find source on "increasing" deployment.

\subsection{Primary Objectives} \label{PrimaryObjectives}
The Primary Objectives of this project are as follows:
\begin{itemize}
    \item An app (for a smartphone) that determines when the user is in an 
    average speed check zone, and begins tracking average speed.
    \item Said app then warns, with an audible warning, if the average speed is 
    above the limit.
    \item Voice commands may be used to launch the app at the user's request 
    (for example, if there is temporary speed check area).
\end{itemize}
\subsection{Secondary Objectives}
If time allows, the project may achieve the following objectives:
\begin{itemize}
    \item The app allows for the user to manually set what the speed limit is 
    (for example if a road has a temporarily reduced speed limit).
    \item The app allows for setting the audible warning to a custom sound.
\end{itemize}
\subsection{Tertiary Objectives}
In many territories globally, the use of devices to detect speed cameras is 
illegal, and apps are either explicitly or potentially illegal. It would be 
ideal if the app could detect it was in such a territory, and prevent its own 
usage.
%% source

\section{Scope}
There is scope within the project to provide an app that is cross platform, 
using technologies such as Flutter, UNO, and .NET MAUI (née 
Xamarin.Forms)\footnote{Flutter (citation), UNO (Citation), .NET MAUI 
(Citation).}. This would allow users of the two major mobile ecosystems to use 
the app. However, doing so may hamper the ability of the developer to write 
sufficient code in the time provided. Hence, a cross-platform app is not 
included in the project at this time.

There is also scope to build an independent database of average speed check 
areas. This would be a significant undertaking, and there are already public 
datasets available for use. As such, making such a database is not in the scope 
of this project. However, allowing for contribution to such datasets from 
within the app would serve the public good, and may well be within the scope of 
the project.
% source the datasets. ?

\section{Deliverables}
The Project will deliver an Android Application that meets the Primary 
Objectives (in section \ref{PrimaryObjectives}).

The project will have met its objectives when the app matches provides features 
defined by the Primary Objectives. It will exceed them if it \textbf{also} 
implements Secondary or Tertiary Objectives.

\section{Constraints}
Testing the application in the most \textit{straightforward} way may be 
constrained by applicable law, insurance, and ethics. Otherwise, there are no 
external constraints known at this time.

\section{Assumptions}
There are currently no unknowns for this project; hence, there are no 
assumptions to be made.

\chapter{Project Rationale and Operation}
\section{Project Benefits}
Successful delivery of this project will most benefit drivers (primarily in the 
UK, but possibly in other territories), with phones, who often drive in areas 
with average speed check zones. 

By reducing the time spent monitoring their speedometer, they can be more 
conscious of the road around them, and be more alert to potential incidents. As 
well, by providing \textit{advance} warning of exceeding the average speed, 
they may be able to drive in a smoother manner and avoid the kangaroo-ing" 
effect; this can reduce congestion and prevent accidents.
% Sources, again!
\section{Project Operation}
Agile (source), agile for one (source), milestones and sprints. - Planning 
gannt chart in appendix.
%How will you operate the project?  Will you use a particular methodology for it and for any software development?  How will you measure the success of your choice?
\section{Options}
Kotlin vs Java vs cross-platform. Sources of speed camera data. 
%What options are available to you for the tools, techniques and design parameters of your project?  How will you evaluate them and make the best selection?
\section{Risk Analysis}
Covid (personal, or another lockdown), personal trauma, equipment breakdown, 
etc. - Make a table in an appendix. Mitigation.
%What risks might affect the outcome of your project or its stakeholders?  How severe are they, and what steps will you take to mitigate against them?
\section{Resources Required}
My android device, android studio, anything else?
%What resources will you need for the project?  Are any non-standard?  Are they already available?  What effect will it have if they are not available or are delayed, and how would you manage that?

\end{document}
