\documentclass[11pt, a4paper, notitlepage]{report}
\usepackage[utf8]{inputenc} % Ensures we can use UTF8.
\usepackage[round,colon]{natbib}
\bibliographystyle{hull}
\usepackage{hyperref} % Must come last; allows for links.
\hypersetup{
    hidelinks
}

\title{Smart Kart (Project Initiation Document)}
\date{April 2022}
\author{George Jacob Anthony Kokinis}
\begin{document}
\maketitle
\begin{center}
    Student Number 201910280
    
    Word Count: ?? %exclude acknowledgements, abstract, table of contents, references and appendices) of your document.
\end{center}
\newpage
\section{Abstract}
Average speed check zones (ASC zones), typically enforced using SPECS\footnote{SPECS \citep{specsjenop}} in the UK, are being increasingly deployed in throughout the UK; doubling between 2013 and 2016 \citep{BBCSpeedCameraDoubled}. While useful for enforcing speed limits and increasing safety, with \citet{owenAllsop} finding that fatal and serious collisions dropped by 36.4\% at ASC zones installed purposely to reduce collisions, ASC Zones can lead to distracted driving, as the driver has to monitor their speed, which means looking away from the road to their speedometer for brief periods of time.

This project seeks to create a software application for a smartphone, that detects when a vehicle the phone is in enters an ASC zone, starts tracking the vehicle's speed, and gives the driver an audible alert if their average speed is at risk of breaking the speed limit, so as to reduce dependence on the driver to check their speedometer.

\tableofcontents

\chapter{Introduction}
\section{Context}
Average speed check zones are an alternative to traditional fixed point speed cameras. Fixed point cameras take two photos a given time-delta apart and measure the car's distance using road markings, to calculate the speed: $ speed = distance/time $. This is good for enforcing speed in that one position, but does no enforcement for the road before or after the camera. In contrast, ASC zones effectively use the same methodology but across a longer distance (such as half a mile or 1.5 miles between cameras, and a series of cameras across tens of miles); hence effectively enforcing the speed limit across much larger areas of road.

Currently, there are various products for monitoring a driver's speed and for indicating speed cameras; the main smartphone applications in this space are Google Maps \citep{googleMaps} and Waze \citep{waze}; while Apple Maps \citep{appleMaps} will inform you of fixed speed cameras\footnote{In the UK, these were originally "Gatso" cameras, later followed by Truvelo and Truvelo d-cam \citep{dcam}}, it does not inform you of ASC zones.

However, Google Maps does not register ASCs as actual "zones", but instead as a fixed speed camera at the start of the zone. Waze displays your progression through an ASC, but does not calculate your average speed. TomTom GO Navigation \citep{tomtomGo} does track your average speed in an ASC zone, but operates on a paid subscription model, so is not available to everyone. Hence, there is space in the market for a free solution to monitoring speed in ASC zones.
\section{The Problem}
While lawful drivers should be aware of their speed anyway, it is likely that many check their speed more often and with more discretion whilst within an ASC zone. This means they may be focused on their speedometer when something important is passed, such as a direction sign, a signal from another road user, or an overhead gantry message - such as the "Red X" on Smart Motorways\footnote{Smart Motorways \citep{SmartMotorways}}, or a speed limit change. 
\section{A solution}
By creating a smartphone application that warns a user if they are about to 
exceed the speed limit, the load on the driver can be reduced, allowing them to 
have more awareness of the road. By making this application free to use 
(whether free or supported by advertisements), this increases availability, 
potentially increasing the impact of this project.
\section{Report Structure}
The rest of this report is structured into the following chapters:
\begin{itemize}
	\item Background: Describing and laying out technologies and concepts that 
	may be used in the development of the application or are useful for 
	understanding other technologies.
	\item Aims and Objectives: Specification in detail about what features the 
	application should have and how this will be tested or measured.
	\item Design: A description of the initial design, visual and technical, 
	for the application, and design(s) for surveys to gain feedback on the UI 
	of the application.
	\item Technical Development: Discussion around the development of the application. Note that the actual development log will be listed in Appendix \ref{app:GitLog}.
    \item Evaluation: Discussion and Evaluation of how closely the application meets the Aims and Objectives in the aforementioned chapter.
\end{itemize}
After which there will be Appendices and the Bibliography. % Remove this?

\appendix
\chapter{Development Log}\label{app:GitLog}
TestTestTest

\bibliography{finalrep}
\end{document}