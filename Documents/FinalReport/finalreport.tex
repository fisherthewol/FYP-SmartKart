\documentclass[11pt, a4paper, notitlepage]{report}
\usepackage[utf8]{inputenc} % Ensures we can use UTF8.
\usepackage[round,colon]{natbib}
\bibliographystyle{hull}
\usepackage{amsmath}
\usepackage{hyperref} % Must come last; allows for links.
\hypersetup{
    hidelinks
}

\title{Smart Kart (Project Initiation Document)}
\date{April 2022}
\author{George Jacob Anthony Kokinis}
\begin{document}
\maketitle
\begin{center}
    Student Number 201910280
    
    Word Count: ?? %exclude acknowledgements, abstract, table of contents, references and appendices) of your document.
\end{center}
\newpage
\section{Abstract}
Average speed check zones (ASC zones), typically enforced using SPECS\footnote{SPECS \citep{specsjenop}} in the UK, are being increasingly deployed in throughout the UK; doubling between 2013 and 2016 \citep{BBCSpeedCameraDoubled}. While useful for enforcing speed limits and increasing safety, with \citet{owenAllsop} finding that fatal and serious collisions dropped by 36.4\% at ASC zones installed purposely to reduce collisions, ASC Zones can lead to distracted driving, as the driver has to monitor their speed, which means looking away from the road to their speedometer for brief periods of time.

This project seeks to create a software application for a smartphone, that detects when a vehicle the phone is in enters an ASC zone, starts tracking the vehicle's speed, and gives the driver an audible alert if their average speed is at risk of breaking the speed limit, so as to reduce dependence on the driver to check their speedometer.

\tableofcontents

\chapter{Introduction}
\section{Context}\label{sec:Context}
Average speed check zones are an alternative to traditional fixed point speed cameras. Fixed point cameras take two photos a given time-delta apart and measure the car's distance using road markings, to calculate the speed: $ speed = distance \div time $. This is good for enforcing speed in that one position, but does no enforcement for the road before or after the camera. In contrast, ASC zones effectively use the same methodology but across a longer distance (such as half a mile or 1.5 miles between cameras, and a series of cameras across tens of miles); hence effectively enforcing the speed limit across much larger areas of road.

Currently, there are various products for monitoring a driver's speed and for indicating speed cameras; the main smartphone applications in this space are Google Maps \citep{googleMaps} and Waze \citep{waze}; while Apple Maps \citep{appleMaps} will inform you of fixed speed cameras\footnote{In the UK, these were originally "Gatso" cameras, later followed by Truvelo and Truvelo d-cam \citep{dcam}}, it does not inform you of ASC zones.

However, Google Maps does not register ASCs as actual "zones", but instead as a fixed speed camera at the start of the zone. Waze displays your progression through an ASC, but does not calculate your average speed. TomTom GO Navigation \citep{tomtomGo} does track your average speed in an ASC zone, but operates on a paid subscription model, so is not available to everyone. Hence, there is space in the market for a free solution to monitoring speed in ASC zones.
\section{The Problem}
While lawful drivers should be aware of their speed anyway, it is likely that many check their speed more often and with more discretion whilst within an ASC zone. This means they may be focused on their speedometer when something important is passed, such as a direction sign, a signal from another road user, or an overhead gantry message - such as the "Red X" on Smart Motorways\footnote{Smart Motorways \citep{SmartMotorways}}, or a speed limit change. 
\section{A solution}
By creating a smartphone application that warns a user if they are about to 
exceed the speed limit, the load on the driver can be reduced, allowing them to 
have more awareness of the road. By making this application free to use 
(whether free or supported by advertisements), this increases availability, 
potentially increasing the impact of this project.
\section{Report Structure}
The rest of this report is structured into the following chapters:
\begin{itemize}
	\item Background: Describing and laying out technologies and concepts that may be used in the development of the application or are useful for understanding other technologies.
	\item Aims and Objectives: Specification in detail about what features the application should have and how this will be tested or measured.
	\item Design: A description of the initial design, visual and technical, for the application, and design(s) for surveys to gain feedback on the UI of the application.
	\item Technical Development: Discussion around the development of the application. Note that the actual development log will be listed in Appendix \ref{app:GitLog}.
    \item Evaluation: Discussion and Evaluation of how closely the application meets the Aims and Objectives in the aforementioned chapter.
\end{itemize}
After which there will be Appendices and the Bibliography. % Remove this?
\chapter{Background}
The development of this project requires understanding and examination of various topics across various fields; including Kinematics, Programming, Law, and UX. Hence this chapter discusses and describes relevant topics.
\section{Speed Cameras}
\subsection{Fixed-Point Cameras}
In the United Kingdom, traditional fixed-point speed cameras were of a "Gatso" type, later replaced by "Truvelo", and later Truvelo D-Cam. They are installed either by the Local Authority, the local Police Force, or the relevant highways agency\footnote{National Highways in England, Transport Scotland in Scotland, Traffic Wales in Wales, and DfI Roads in Northern Ireland.}; with all three often forming "Road Safety Partnerships" for given areas, that can then receive grants from the central government \citep{RSPGrantDetails} to use for, among other things, the installation of speed cameras. The decision of where to install a speed camera is made on a few factors; with \citet{SpeedCameraInstalltion} claiming that at the proposed location, greater than 20\% of drivers must exceed the speed limit, and that there must be a history of serious accidents.

At first glance, fixed-point cameras appear to work on a rather simple principle. As described in the \nameref{sec:Context}, the simplest view of how to obtain the speed of a vehicle is $ speed = distance~\div~time $. Hence the camera can take two measurements a known time apart, work out the distance the vehicle travels between those measurements, and calculate the speed. 
Fixed-point cameras use K and Ku-band Radar signals to determine the speed of the vehicle; K-band meaning that the frequency used is between 18 and 27 GHz \citep{IEEERadar}; and Radar, an acronym for Radio Detection and Ranging \citep{RadarNaming}, referring to the usage of a transmitter, receiver, and processing of K-Band or adjacent frequencies (in the Radio or Microwave ranges) to determine properties of an object. The time of flight is the total time between transmitting a signal and receiving the reflection; and this can be used to determine the distance. The speed of light is known\footnote{299 792 458 metres per second in a vacuum, known as \textit{c}, and slower in air: which can be calculated with $ speed = c \div n $ \citep{HechtOptics}; where n is the refractive index of the medium}, and so the distance can be calculated using a rearrangement of the formula: $ distance = speed \times time $. Hence, overall, the speed of a vehicle travelling away from camera can be calculated by:
\begin{equation}
	first~measurement = \frac{c}{1.0003}\footnote{1.0003 is the approximate refractive index of light in Air \citep{refIndxAir}} \times time~of~flight
\end{equation}
\begin{equation}
	second~measurement = \frac{c}{1.0003} \times time~of~flight
\end{equation}
\begin{equation}
	speed = \frac{second~measurement~-~first~measurement}{time~between~measurements}
\end{equation}
However, this requires storage and memory of 

\appendix
\chapter{Development Log}\label{app:GitLog}
TestTestTest

\bibliography{finalrep}
\end{document}